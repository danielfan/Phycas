\documentclass[12pt]{article}
\setlength{\oddsidemargin}{0.in}
\setlength{\textwidth}{6.5in}
\setlength{\topmargin}{-0.25in}
\setlength{\textheight}{8.25in}

% Skip space between paragraphs
\setlength{\parskip}{.1in}

% Do not put numbers in section headings
\setcounter{secnumdepth}{0}

% Indent paragraphs 0. in
\setlength{\parindent}{0.in}

% Use the natbib package for the bibliography
\usepackage[round]{natbib}
\bibliographystyle{sysbio}

% Use the graphicx package to incorporate and scale
% encapsulated postscript figures
\usepackage{graphicx}

% Make document single-spaced
\renewcommand{\baselinestretch}{1.0}

\newcommand{\hppfile}[1]{{\tt #1}}
\newcommand{\cppfile}[1]{{\tt #1}}
\newcommand{\newterm}[1]{{\bfseries #1}}
\newcommand{\function}[1]{{\tt #1}}
\newcommand{\datamember}[1]{{\tt #1}}

\begin{document}

\title{From nexus to tree: how Phycas handles observed data}
\author{Paul O. Lewis}
\date{7 January 2008}
\maketitle

This document describes what is involved with storing data (originating in a nexus data file) in the TipData structures associated with TreeNode objects in a Tree. This process involves representing a variety of discrete data types (nucleotide, protein, discrete morphological characters) in a uniform way, handling missing and ambiguous states, and compression of simple data patterns.

\section{The TipData class}

The TipData class (\hppfile{tip\_data.hpp}) stores the data associated with each tip in the tree. The description below assumes the data was
read from the following nexus data file:
\begin{verbatim}
#nexus

begin data;
  dimensions ntax=4 nchar=5;
  format datatype=dna missing=? gap=-;
  matrix
    taxon1 A ? T  G    {CGT}
    taxon2 A C - {ACG}  T 
    taxon3 A C -  N    (AG)
    taxon4 A C T  R     Y
  ;
end;
\end{verbatim}

The above nexus file would be stored internally (i.e., CipresNative::DiscreteMatrix) in the following form, where 
states or state combinations have been replaced with integers ranging from -1 to 9:

\begin{verbatim}
taxon1  0 4  3 2 6
taxon2  0 1 -1 7 3
taxon3  0 1 -1 5 8
taxon4  0 1  3 8 9
\end{verbatim}

The \newterm{global state list position vector} (i.e. `CipresNative::DiscreteMatrix::getStateListPos()') is 
\begin{verbatim}
  0 2 4 6 8 14 19 23 27 30
\end{verbatim}

Each of the above values represents an index into the \newterm{global state list}, which looks like this:
\begin{verbatim}
  1 0 1 1 1 2 1 3 5 -1 0 1 2 3 4 0 1 2 3 3 1 2 3 3 0 2 3 2 0 2 4 1 3
\end{verbatim}

The table below explains this enigmatic global state list. The asterisks in the first column mark the elements of 
the global state list position vector given above. The second column is the global state list, which is also given 
above. The third column contains the codes used for states internally. Finally, the last column relates	these global 
state codes to the representation in the original nexus file. Horizontal lines mark the boundaries between states. 
The first state list element in each section holds the number of subsequent state list elements that need to be read
in order to fully characterize the state or state combination. The unambiguous states come first, followed by the 
state code representing complete ambiguity (corresponding to the missing data symbol (usually '?') in the nexus data
file). After that, ambiguous states are added as they are encountered in the data file (e.g., the next global state
code is 5, corresponding to 'N' in the nexus file, which means "A, C, G or T"). The gap (or inapplicable) state is 
always represented by -1 and has no separate entry in the global state list.

\begin{verbatim}
          global   global     representation
          state    state      in nexus file
index     list     code  -->  ambig.nex
---------------------------------------------
  0*        1       0    -->  A
  1         0
---------------------------------------------
  2*        1       1    -->  C
  3         1
---------------------------------------------
  4*        1       2    -->  G
  5         2
---------------------------------------------
  6*        1       3    -->  T
  7         3
---------------------------------------------
  8*        5       4    -->  ?
  9        -1                 
 10         0       Note that ? allows gaps
 11         1       in addition to A, C, G or
 12         2       T, so it is even more
 13         3       ambiguous than N
---------------------------------------------
 14*        4       5    -->  N
 15         0
 16         1
 17         2
 18         3
---------------------------------------------
 19*        3       6    -->  {CGT}
 20         1
 21         2
 22         3
---------------------------------------------
 23*        3       7    -->  {ACG}
 24         0
 25         2
 26         3
---------------------------------------------
 27*        2       8    -->  R, {AG}
 28         0
 29         2
---------------------------------------------
 30*        4       9    -->  Y
 31         1
 32         3
---------------------------------------------
\end{verbatim}
	
For large datasets with many different ambiguity combinations, the global state list could grow quite large. Each 
state code ends up being a row in the augmented transposed transition probability matrix used in computing 
conditional likelihood arrays, and it could be quite inefficient if these augmented matrices had many unnecessary 
rows. Thus, when a row in the global matrix is copied to the tip of a tree, where it is used to compute likelihoods,
a translation is done into \newterm{local state codes}. The local codes are identical to the global codes for the primary states and the state representing complete ambiguity, but may differ from the global codes for codes representing partial ambiguities. To illustrate, consider taxon4 in the example. It has these states in the nexus data file:
\begin{verbatim}
  A  C  T  R  Y
\end{verbatim}

which translate to these global state codes:
\begin{verbatim}
  0  1  3  8  9
\end{verbatim}

When copied to a tip node in a tree, however, these codes become translated to
\begin{verbatim}
  0  1  3  5  6
\end{verbatim}

Global codes 8 and 9 have become 5 and 6, respectively.

Note that the constructor is private and thus TipData objects can only be created by the friend function
\function{TreeLikelihood::allocateTipData}.

\section{The TreeLikelihood::allocateTipData function}

The \function{allocateTipData} function (\cppfile{tree\_likelihood.cpp}) allocates the TipData structure needed to store the data for one tip (the tip corresponding to the supplied row index). The function returns a pointer to the newly-created TipData structure. This function performs the translation of global state codes to local state codes.

\section{The TreeLikelihood::compressDataMatrix function}

The \function{compressDataMatrix} function (\cppfile{tree\_likelihood.cpp}) is responsible for compressing the data patterns. This function takes a reference to a CipresNative::DiscreteMatrix object and fills data members \datamember{pattern\_map}, \datamember{pattern\_counts} and \datamember{charIndexToPatternIndex}.

\section{The TreeNode::TipDataDeleter function pointer}

The \function{TipDataDeleter} function is responsible for deleting the TipData structure allocated by the \function{TreeLikelihood::allocateTipData} function. The function \function{TreeNode::SetTipData} is used to set both \datamember{tipData} and \datamember{tipDataDeleter}.

%
% Figure "felsenstein"
%
%\clearpage
%\begin{figure}
%\centering
%\hfil\includegraphics[scale=0.7]{felsenzone.eps}\hfil
%\end{figure}

%\section{References}
%\renewcommand{\bibsection}{}
%\bibliography{example}

\end{document}
