\documentclass{article}

\setcounter{page}{1}
\setcounter{secnumdepth}{0}
\setlength{\oddsidemargin}{0.in}
\setlength{\evensidemargin}{0.in}
\setlength{\textwidth}{6.5in}
\setlength{\topmargin}{0.0in}
\setlength{\headheight}{0.17in}
\setlength{\headsep}{0.35in}
\setlength{\textheight}{8.in}
\setlength{\parindent}{0in}
\setlength{\parskip}{.1in}
%\input seteps
\usepackage[round]{natbib}
\bibliographystyle{sysbio}

\usepackage{xspace}
\newcommand{\barLone}{\bar{L}_1}
\newcommand{\barLzero}{\bar{L}_0}
\newcommand{\execmd}[1]{\texttt{#1}\\}
\newcommand{\cmdopt}[1]{\texttt{#1}\xspace}
\newcommand{\cmd}[1]{\texttt{#1}\xspace}
\newcommand{\mb}{MrBayes\xspace}
\newcommand{\paup}{PAUP*\xspace}
\newcommand{\phycas}{Phycas\xspace}
\newcommand{\localfile}[1]{\textsf{#1}\xspace}
\newcommand{\QandA}[2]{\textit{#1}\footnote{#2}\xspace}
% use the fancyhdr package to allow both headers and footers
\usepackage{fancyhdr}
\pagestyle{fancy}
\fancyhf{} % clear all headers and footers
\fancyhead[LO,RE]{Woods Hole}
\fancyhead[RO,LE]{Workshop on Molecular Evolution}
\fancyfoot[LO,LE]{{\footnotesize \copyright\ 2008 by Paul O. Lewis, Mark T. Holder, and David Swofford}}
\fancyfoot[CE,CO]{}
\fancyfoot[RO,RE]{\thepage}

\usepackage{url}

\usepackage{hyperref}
\hypersetup{backref,  pdfpagemode=FullScreen,  linkcolor=blue, citecolor=red, colorlinks=true, hyperindex=true}
\fancyhead[CE,CO]{\phycas Tutorial}
\begin{document}

%###########################################################################################
\section{\phycas Lab}

%###########################################################################################

The main goal is to become familiar with running \phycas and interpreting its output. 
Don't rush things. 
If you don't finish the lab, you can always download 
\phycas from \url{http://www.phycas.org} when you get home and finish it there.

Stylistic conventions used in this document:

\begin{tabular}{l}
	Commands understood by the \paup and/or \mb programs are in \cmd{this fixed width font} \\
	Questions for you to answer are in {\em italics}. \\
	Important things to note are in {\bf bold face}. \\
	Names of files \localfile{are in  this font}.\\
	Web site URLs look \url{http://like_this} and are clickable links.
\end{tabular}

%-------------------------------------------------------------------------------------------
\subsection{Background}
%-------------------------------------------------------------------------------------------
\phycas is a software package for Bayesian phylogenetic analysis written by Paul Lewis with
some contributions by Mark Holder and David Swofford.
\phycas is a hybrid program: computationally-intensive tasks are implemented in C++, and the high-level
tasks are done in \href{http://www.python.org/}{python}\footnote{The Phycas manual contains instructions for installing python if you are using an operating system
that does not come with python installed.}.

In fact, the python language is the front-end for \phycas. 
So when you write command file in \phycas you are actually writing a python program.
This has advantages because python interpreter is a powerful, robust, and versatile program for interpreting text commands.
However, it also means that \phycas cannot be forgiving when it comes to the structure of commands and syntax in general.

\subsubsection{The Phycas.app bundle}
In the lab we will be running \phycas as a double-clickable program that is built around the open-source iTerm Terminal application (\href{http://iterm.sourceforge.net/}{iTerm homepage}).
You can run phycas scripts from any terminal by invoking python and entering the command:
\begin{verbatim}
from phycas import *
\end{verbatim}
You will not have to issue this command when using the \phycas application bundle on Mac, but you will have to use the command in all of your phycas scripts.
\subsubsection{IPython}
In the lab we will be running \phycas through \href{http://ipython.scipy.org/moin/}{IPython} rather than Python itself.  
IPython provides a more user-friendly environment for interactive python than the ``raw'' python interpreter.
The main benefits of IPython are:
\begin{itemize}
	\item Tab-completion of variable names (start to type the name and then hit the Tab-key to fill in the rest of the word or see a list of possible matches),
	\item Native support for a few UNIX commands (most notably \cmd{ls} and \cmd{cd})
	\item Support for any UNIX commands if you start the line with an exclamation point (!)
\end{itemize}
If you decide that you do not like IPython, then you can simply edit the active configuration file of phycas. 
Open 
\localfile{$\sim$/.phycas/active\_phycas\_env.sh} 
in a text editor, and change the last line to read:
\begin{verbatim}
export USES_I_PYTHON=0
\end{verbatim}

If you are unsure whether or not you have correctly enabled (or disabled) the IPython-style of interacting with \phycas, you can tell by looking at the prompt.
The IPython prompt looks like this:
\begin{verbatim}
In [2]: 
\end{verbatim}
while the python prompt looks like this:
\begin{verbatim}
>>>
\end{verbatim}


\subsection{Exercise 1}
Locate the directory of example files labeled \localfile{PhycasWHExamples} and move it to a convenient location for you.

You can launch the \phycas application you can double-click the \phycas icon application (later when you have scripts written, you can launch the \phycas by dragging a python script onto the \phycas application icon).
After the libraries have loaded you will see a prompt.
This is the python interpreter waiting for you to give it a command.

\subsubsection{Changing the working directory}
The first step is to set the working directory of \phycas to the directory in which you want to work.
If you are using the IPython interface, and you would like to  move to a directory called \localfile{Desktop/PhycasWHExamples} then issue the command:
\begin{verbatim}
cd Desktop/PhycasWHExamples
pwd
ls
\end{verbatim}
to change working directory, then show the path to the current directory

Note that, if you are using the  python interpreter (instead of the IPython interface), you have to issue the commands:
\begin{verbatim}
import os
os.chdir("Desktop/PhycasWHExamples")
from phycas import *
\end{verbatim}

\subsubsection{Getting help}
A good first step is to get an idea of what actions you can take. 
Call the help function to see a general message about \phycas, and then
use the \cmd{public} function to see a summary of globally available
variables
\begin{verbatim}
help()
public()
\end{verbatim}

Because you are actually programming python with every line you type, you 
also have access to all python commands.
You can ask python for all of the variables that are in memory using the
\cmd{dir} function:
\begin{verbatim}
dir()
\end{verbatim}
This will return a (long) list of strings that are the names of all of the currently
available names that python knows about.  
You can ignore any name that starts with an underscore.

Well, a very long list of variables is probably not too helpful to you.
Fortunately, you can pass one of the listed names in the help function to see 
some information about the variable:
\begin{verbatim}
help(readFile)
help(mcmc)
\end{verbatim}

\subsubsection{Setting up a basic MCMC analysis}
The settings that affect the behavior of an MCMC run are all stored
as attributes of an \phycas object with the name \cmd{mcmc}.
In python, objects are collections of data and actions.
You address the attributes of an object by putting a ``.'' between
the name of the object and the name of the attribute.
And you change the value of an attribute by using the ``='' character
to assign the attribute a new value
Let's see how it works:
\begin{verbatim}
mcmc.current()
mcmc.data_source = "green.nex"
mcmc.current()
\end{verbatim}
If you look through the tables of input settings before and after
the assignment to the \cmdopt{data\_source} attribute, you should see
that the value goes from ``\cmd{None}'' to ``\cmd{Characters from the file green.nex}''

As with \mb\citep{MRBAYES}, the MCMC sampler will produce a collection of files storing
trees and parameter values.
Now lets configure the output settings:
\begin{verbatim}
mcmc.out.log = "basic.log"
mcmc.out.log.mode = REPLACE
mcmc.out.trees.prefix = "green"
mcmc.out.params.prefix = "green"
\end{verbatim}

Now we will set up the length of the MCMC simulation. 
It is important to realize that a ``cycle'' in \phycas is {\bf not}
equivalent to a ``generation'' in \mb.
In \mb each generation represents an iteration of:
\begin{enumerate}
	\item propose a new state,
	\item evaluate the posterior probability density,
	\item accept or reject the proposal
\end{enumerate}
But in \phycas a cycle represents a sweep of proposals over all of the parameters
in model -- including several proposed changes to the topology.
In terms of the number of updates, a cycle in \phycas is more or less equivalent (in terms 
of the number of updates) to 100 generations in \mb:
\begin{verbatim}
mcmc.ncycles = 2000
mcmc.sample_every = 10
\end{verbatim}

Before we launch the analysis, let's check the settings according to the MCMC command
to see if there were any errors in the settings:
\begin{verbatim}
mcmc.current()
\end{verbatim}

If everything looks good we can launch the run by calling the \cmd{mcmc} command
as if it were a python function:
\begin{verbatim}
mcmc()
\end{verbatim}

As in \mb, we can summarize the tree trees produced by the \cmd{mcmc} analysis
using a \cmd{sumt} command.
Unlike \mb, the \cmd{sumt} command is not automatically configured to read the files
that were just created.
Once again, we can use the \cmd{current} function to inspect a command's state.
Let's tell the \cmd{sumt} command to read the trees from the \localfile{green.t} file
that was just created by the MCMC analysis:
\begin{verbatim}
sumt.current()
sumt.trees = "green.t"
sumt.current()
\end{verbatim}

Then we can rune the sumt command with the syntax:
\begin{verbatim}
sumt()
\end{verbatim}

You will see the majority-rule consensus trees and the trees that have the 
highest posterior probability in the \localfile{sumt\_trees.pdf} file.
In the case of this simple dataset and this short MCMC, it is entirely possible that all of the 
splits will have estimated posterior probabilities of 1.0.
If that is the case, the \cmd{sumt} command will not produce a pdf file showing 
plots of the posterior probability over different divisions of the MCMC
simulation.
We will see examples of these files later in the lab.

The commands for replaying this example are shown in the section \hyperref[basicpy]{basic.py} and in the file \localfile{basic.py} in the scripts folder that accompanied the lab.

Actually a slightly longer, but more 
\section{Polytomy priors}
One flavor of Bayesian analysis that is implemented in \phycas, but not available in \mb is support for prior distributions that place non-zero probability on trees that are not fully-resolved (also referred to as trees with polytomies or trees with multifurcations).
The details of the algorithm (and justifications for the approach) can be found in \citet{LewisHH2005}.
We will be using the same data set that was used in that paper which is a set of algal sequences published by \citet{ShoupL2003}.

To read in the data into memory we invoke the \cmd{readFile} function.
We pass in the full path to the NEXUS file that we would like to read as an argument to the function, and we capture a reference to the data that has been read by storing the return value of the function.
If we would like to execute the example data file ``green.nex'' and store the results in a variable called 
\cmdopt{file\_contents} then we use the command:
\begin{verbatim}
file_contents = readFile("green.nex")
\end{verbatim}
At this point you should be able to inspect the data that was read. 
The easiest way to figure out about an object is to ask for help:
\begin{verbatim}
help
help(file_contents)
\end{verbatim}
The first invocation of help shows the general message and the second form is how you ask for help about a specific object.

\begin{verbatim}
mcmc.data_source = 'green.nex' #file_contents.characters
mcmc.out.log = 'basic.log'
mcmc.out.log.mode = REPLACE
mcmc.out.trees.prefix = 'green'
mcmc.out.params.prefix = 'green'
mcmc.ncycles = 2000
mcmc.sample_every = 10
mcmc()
\end{verbatim}


\newpage
\subsection{basic.py}\label{basicpy}
\begin{verbatim}
#!/usr/bin/env python
from phycas import *

mcmc.current()
mcmc.data_source = "green.nex"
mcmc.out.log = "basic.log"
mcmc.out.log.mode = REPLACE
mcmc.out.trees.prefix = "green"
mcmc.out.params.prefix = "green"
mcmc.ncycles = 2000
mcmc.sample_every = 10
mcmc()
sumt.trees = "green.t"
sumt()
\end{verbatim}



\newpage
\bibliography{phylo}
\end{document}

\label{executingdatafile}
\execmd{cmdhere}
\localfile{algaemb.nex} 
\cmdopt{tratiopr} 

\QandA{q?}{answer}


