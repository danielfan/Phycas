\documentclass{article}

\setcounter{page}{1}
\setcounter{secnumdepth}{0}
\setlength{\oddsidemargin}{0.in}
\setlength{\evensidemargin}{0.in}
\setlength{\textwidth}{6.5in}
\setlength{\topmargin}{0.0in}
\setlength{\headheight}{0.17in}
\setlength{\headsep}{0.35in}
\setlength{\textheight}{8.in}
\setlength{\parindent}{0in}
\setlength{\parskip}{.1in}
%\input seteps
\usepackage[round]{natbib}
\bibliographystyle{sysbio}

\usepackage{xspace}
\newcommand{\barLone}{\bar{L}_1}
\newcommand{\barLzero}{\bar{L}_0}
\newcommand{\execmd}[1]{\texttt{#1}\\}
\newcommand{\cmdopt}[1]{\texttt{#1}\xspace}
\newcommand{\cmd}[1]{\texttt{#1}\xspace}
\newcommand{\mb}{MrBayes\xspace}
\newcommand{\paup}{PAUP*\xspace}
\newcommand{\phycas}{Phycas\xspace}
\newcommand{\localfile}[1]{\textsf{#1}\xspace}
\newcommand{\QandA}[2]{\textit{#1}\footnote{#2}\xspace}
% use the fancyhdr package to allow both headers and footers
\usepackage{fancyhdr}
\pagestyle{fancy}
\fancyhf{} % clear all headers and footers
\fancyhead[LO,RE]{Phycas demo}
\fancyhead[RO,LE]{2008 Workshop on Molecular Evolution Woods Hole}
\fancyfoot[LO,LE]{{\footnotesize \copyright\ 2008 by Paul O. Lewis, Mark T. Holder, and David Swofford}}
\fancyfoot[CE,CO]{}
\fancyfoot[RO,RE]{\thepage}

\usepackage{url}

\usepackage{hyperref}
\hypersetup{backref,  pdfpagemode=FullScreen,  linkcolor=blue, citecolor=red, colorlinks=true, hyperindex=true}
\begin{document}

%###########################################################################################
\section{\phycas Lab}

%###########################################################################################

The main goal is to become familiar with running \phycas and interpreting its output. 
Don't rush things. 
If you don't finish the lab, you can always download 
\phycas from \url{http://www.phycas.org} when you get home and finish it there.

Stylistic conventions used in this document:

\begin{tabular}{l}
	Commands understood by the \phycas are in \cmd{this fixed width font} \\
	Questions for you to answer are in {\em italics}. \\
	Important things to note are in {\bf bold face}. \\
	Names of files \localfile{are in  this font}.\\
	Web site URLs look \url{http://like_this} and are clickable links.
\end{tabular}

%-------------------------------------------------------------------------------------------
\subsection{Background}
%-------------------------------------------------------------------------------------------
\phycas is a software package for Bayesian phylogenetic analysis written by Paul Lewis with
some contributions by Mark Holder and David Swofford.
\phycas is a hybrid program: computationally-intensive tasks are implemented in C++, and the high-level
tasks are done in \href{http://www.python.org/}{python}\footnote{The Phycas manual contains instructions for installing python if you are using an operating system
that does not come with python installed.}.

In fact, the python language is the front-end for \phycas. 
So when you write command file in \phycas you are actually writing a python program.
This has advantages because python interpreter is a powerful, robust, and versatile program for interpreting text commands.
However, it also means that \phycas cannot be forgiving when it comes to the structure of commands and syntax in general.

\subsubsection{The Phycas.app bundle}
In the lab we will be running \phycas as a double-clickable program that is built around the open-source iTerm Terminal application (\href{http://iterm.sourceforge.net/}{iTerm homepage}).
You can run phycas scripts from any terminal by invoking python and entering the command:
\begin{verbatim}
from phycas import *
\end{verbatim}
You will not have to issue this command when using the \phycas application bundle on Mac, but you will have to use the command in all of your phycas scripts.
\subsubsection{IPython}
In the lab we will be running \phycas through \href{http://ipython.scipy.org/moin/}{IPython} rather than Python itself.  
IPython provides a more user-friendly environment for interactive python than the ``raw'' python interpreter.
The main benefits of IPython are:
\begin{itemize}
	\item Tab-completion of variable names (start to type the name and then hit the Tab-key to fill in the rest of the word or see a list of possible matches),
	\item Native support for a few UNIX commands (most notably \cmd{ls} and \cmd{cd})
	\item Support for any UNIX commands if you start the line with an exclamation point (!)
\end{itemize}
If you decide that you do not like IPython, then you can simply edit the active configuration file of phycas. 
Open 
\localfile{$\sim$/.phycas/active\_phycas\_env.sh} 
in a text editor, and change the last line to read:
\begin{verbatim}
export USES_I_PYTHON=0
\end{verbatim}

If you are unsure whether or not you have correctly enabled (or disabled) the IPython-style of interacting with \phycas, you can tell by looking at the prompt.
The IPython prompt looks like this:
\begin{verbatim}
In [2]: 
\end{verbatim}
while the python prompt looks like this:
\begin{verbatim}
>>>
\end{verbatim}


\subsection{Exercise 1}
Locate the directory of example files labeled \localfile{PhycasWHExamples} and move it to a convenient location for you.

You can launch the \phycas application you can double-click the \phycas icon application (later when you have scripts written, you can launch the \phycas by dragging a python script onto the \phycas application icon).
After the libraries have loaded you will see a prompt.
This is the python interpreter waiting for you to give it a command.

\subsubsection{Changing the working directory}
The first step is to set the working directory of \phycas to the directory in which you want to work.
If you are using the IPython interface, and you would like to  move to a directory called \localfile{Desktop/PhycasWHExamples} then issue the command:
\begin{verbatim}
cd Desktop/PhycasWHExamples
pwd
ls
\end{verbatim}
to change working directory, then show the path to the current directory

Note that, if you are using the  python interpreter (instead of the IPython interface), you have to issue the commands:
\begin{verbatim}
import os
os.chdir("Desktop/PhycasWHExamples")
from phycas import *
\end{verbatim}

\subsubsection{Getting help}
A good first step is to get an idea of what actions you can take. 
Call the help function to see a general message about \phycas, and then
use the \cmd{public} function to see a summary of globally available
variables
\begin{verbatim}
help()
public()
\end{verbatim}

Because you are actually programming python with every line you type, you 
also have access to all python commands.
You can ask python for all of the variables that are in memory using the
\cmd{dir} function:
\begin{verbatim}
dir()
\end{verbatim}
This will return a (long) list of strings that are the names of all of the currently
available names that python knows about.  
You can ignore any name that starts with an underscore.

Well, a very long list of variables is probably not too helpful to you.
Fortunately, you can pass one of the listed names in the help function to see 
some information about the variable:
\begin{verbatim}
help(readFile)
help(mcmc)
\end{verbatim}

\subsubsection{Setting up a basic MCMC analysis}
The settings that affect the behavior of an MCMC run are all stored
as attributes of an \phycas object with the name \cmd{mcmc}.
In python, objects are collections of data and actions.
You address the attributes of an object by putting a ``.'' between
the name of the object and the name of the attribute.
And you change the value of an attribute by using the ``='' character
to assign the attribute a new value
Let's see how it works:
\begin{verbatim}
mcmc.current()
mcmc.data_source = "green.nex"
mcmc.current()
\end{verbatim}
If you look through the tables of input settings before and after
the assignment to the \cmdopt{data\_source} attribute, you should see
that the value goes from ``\cmd{None}'' to ``\cmd{Characters from the file green.nex}''

As with \mb\citep{RonquistH2003}, the MCMC sampler will produce a collection of files storing
trees and parameter values.
Now lets configure the output settings:
\begin{verbatim}
mcmc.out.log = "basic.log"
mcmc.out.log.mode = REPLACE
mcmc.out.trees.prefix = "green"
mcmc.out.params.prefix = "green"
\end{verbatim}

Now we will set up the length of the MCMC simulation. 
It is important to realize that a ``cycle'' in \phycas is {\bf not}
equivalent to a ``generation'' in \mb.
In \mb each generation represents an iteration of:
\begin{enumerate}
	\item propose a new state,
	\item evaluate the posterior probability density,
	\item accept or reject the proposal
\end{enumerate}
But in \phycas a cycle represents a sweep of proposals over all of the parameters
in model -- including several proposed changes to the topology.
In terms of the number of updates, a cycle in \phycas is more or less equivalent (in terms 
of the number of updates) to 100 generations in \mb:
\begin{verbatim}
mcmc.ncycles = 2000
mcmc.sample_every = 10
\end{verbatim}

Before we launch the analysis, let's check the settings according to the MCMC command
to see if there were any errors in the settings:
\begin{verbatim}
mcmc.current()
\end{verbatim}

If everything looks good we can launch the run by calling the \cmd{mcmc} command
as if it were a python function:
\begin{verbatim}
mcmc()
\end{verbatim}

As in \mb, we can summarize the tree trees produced by the \cmd{mcmc} analysis
using a \cmd{sumt} command.
Unlike \mb, the \cmd{sumt} command is not automatically configured to read the files
that were just created.
Once again, we can use the \cmd{current} function to inspect a command's state.
Let's tell the \cmd{sumt} command to read the trees from the \localfile{green.t} file
that was just created by the MCMC analysis:
\begin{verbatim}
sumt.current()
sumt.trees = "green.t"
sumt.current()
\end{verbatim}

Then we can run the sumt command with the syntax:
\begin{verbatim}
sumt()
\end{verbatim}
\textbf{The sumt.burnin setting in \phycas is specified in terms of the number of trees to skip in the file -- not in terms of the generation number.}
You will see the majority-rule consensus trees and the trees that have the 
highest posterior probability in the \localfile{sumt\_trees.pdf} file.
In the case of this simple dataset and this short MCMC, it is entirely possible that all of the 
splits will have estimated posterior probabilities of 1.0.
If that is the case, the \cmd{sumt} command will not produce a pdf file showing 
plots of the posterior probability over different divisions of the MCMC
simulation.
We will see examples of these files later in the lab.

The commands for replaying this example are shown in the section \hyperref[basicpy]{basic.py} and in the file \localfile{basic.py} in the scripts folder that accompanied the lab.

\subsubsection{Another way to read in data}
A slightly longer, but more more explicit way to have written the previous 
script would have started with the lines:
\begin{verbatim}
file_contents = readFile("green.nex")
mcmc.data_source = file_contents.characters
\end{verbatim}

In this form we read in the data before telling the \cmd{mcmc} to use this data.
So after we read the file, we can inspect the contents:
\begin{verbatim}
help(file_contents)
print file_contents.taxon_labels
print file_contents.characters
print file_contents.trees
\end{verbatim}



\section{Polytomy priors}
One flavor of Bayesian analysis that is implemented in \phycas, but not available in \mb is support for prior distributions that place non-zero probability on trees that are not fully-resolved (also referred to as trees with polytomies or trees with multifurcations).
The details of the algorithm (and justifications for the approach) can be found in \citet{LewisHH2005}.
We will be using the same data set that was used in that paper which is a set of algal sequences published by \citet{ShoupL2003}.

This time we will pay more attention to the model and mcmc settings than we did in the basic example above.
To start from a clean slate:
\begin{enumerate}
	\item open a new window in the \phycas application
	\item change the working directory to a convenient location for working with the \localfile{ShoupLewis.nex}. Note that you do {\bf not} have to change your working directory to match the location to the input file, but if you do not then you will have to make sure that you use a relative or absolute path to all of the files that you refer to.
\end{enumerate}

Read the file into memory:
\begin{verbatim}
file_contents = readFile("ShoupLewis.nex")
mcmc.data_source = file_contents.characters
\end{verbatim}
if you print the \cmd{file\_contents.character} attribute you should see that the matrix has 17 taxa and 3341 characters.

\subsection{Setting up the model of evolution}
Now we will configure the model of character evolution. 
Use the \cmd{model.help()} invocation to see the current settings of the model and the explanations of the
attributes.
Note that the \cmd{model.type} is a string that can be \texttt{"gtr"}, \texttt{"hky"} or \texttt{"jc"} and this controls whether some of the other settings are used (for instance the settings associated with kappa are only used if the model.type is \texttt{"hky"}).

Let's configure \phycas to use the HKY model with $\Gamma$-distributed rate heterogeneity:

\begin{verbatim}
model.type = 'hky'
model.num_rates = 4
\end{verbatim}
the \cmd{model.num\_rates} is the number of variable rates that are used to approximate the Gamma distribution. 
If you were to set \cmd{model.num\_rates = 1} then you would be selecting a single-rate model that does not use $\Gamma$-distributed rate heterogeneity at all.
We can set the value of the shape parameter of the Gamma distribution, but this is just the initial value that the sampler will start at.  
It is more important (in the context of setting up an MCMC run) to set the prior distribution for the gamma's shape parameter.
We can do this by choosing from among the Continuous probability distributions over non-negative numbers.
An Exponential distribution will be fine for the purpose of the lab (The Gamma and Inverse-Gamma distributions are also possible choices):
\begin{verbatim}
model.gamma_shape_prior = Exponential(2.0)
\end{verbatim}
This has just set the prior to be an exponential distribution with mean of 0.5.
\\{\bf NOTE:}  \phycas parameterizes the exponential distribution with the rate parameter with is the reciprocal of the mean of the distribution!
If you are unsure, you can always ``ask'' a distribution what it's mean is:
\begin{verbatim}
model.gamma_shape_prior.getMean()
\end{verbatim}


Kappa ($\kappa$), the transition/transversion rate ratio, is usually much greater than 1, so let's use a fairly vague prior with a mean of 4:
\begin{verbatim}
model.kappa_prior = Exponential(0.25)
\end{verbatim}

The only other parameters in the HKY substitution model are the base frequencies\footnote{
Internally, \phycas uses parameters that are unnormalized $(0,\infty)$ parameters.
The base frequencies are obtained by dividing each of the parameters by the sum of the four.
We place priors on the unnormalized versions, which is why we use a GammaDistribution in this example.
Note that putting Gamma($a,1$) priors on the unnormalized versions of the parameters is 
equivalent to placing a Dirichlet($a,a,a,a$) on the frequencies themselves.
}:
\begin{verbatim}
model.base_freq_param_prior = Gamma(1.0, 1.0)
\end{verbatim}

\subsubsection{The priors on the tree}
The tree and branch lengths are also important parts of our model and they need to have
priors assigned to them.
Currently \phycas only support a uniform priors over all tree topologies (within a
class of trees with the same number of internal nodes); so we do not have to set
a prior on the shapes of different resolved trees.
However we do need to say something about the branch lengths.

We could use an Exponential (or Gamma distribution) for the prior on all of the branch lengths, but then we would have to choose a mean.
Choosing a reasonable prior is important because:
\begin{enumerate}
	\item  estimating branch lengths accurately is often crucial to getting the topology correct,
	\item the prior will be applied to a large number of parameters (because there are so many branches in the tree).
\end{enumerate}
\citet{SuchardWS2001} suggested the use of a hierarchical model in which the branch lengths are distributed as exponential variables with a mean of $\mu$, where $\mu$ is a hyperparameter that 
has a prior distribution set by the investigator.
This has the advantage of producing a ``fat-tailed'' prior distribution on the branch length in which the 
mean of the prior distribution for the branch lengths is allowed to tune-itself to the data at hand.
In \phycas, if you have the \cmd{model.edgelen\_hyperprior} attribute set to a distribution (rather than \texttt{None}), you will be using this hierarchical model suggested by \citet{SuchardWS2001}.
For example:
\begin{verbatim}
model.edgelen_hyperprior = InverseGamma(2.10000, 0.90909)
\end{verbatim}
uses an Inverse-Gamma distribution with mean 1 and variance 10 for the hyperparameter, $\mu$.

\subsubsection{Allowing Polytomies -- or Not}
Let's perform an initial run in the ``standard'' mode in which polytomies are not considered.
We tell the \cmd{mcmc} command that we do not want to sample trees with polytomies:
\begin{verbatim}
mcmc.allow_polytomies = False
\end{verbatim}

Because we are going to perform an analysis that uses a pseudorandom number generator, it is a 
good idea to explicitly set the seed so that we can repeat the analysis.
Let's create a random number generator, set it's seed, and assign it to the mcmc command, and 
tell the mcmc command to start from a random tree that is generated using the same random
number generator:

\begin{verbatim}
rng = Lot()
rng.setSeed(13957)
mcmc.rng = rng
mcmc.starting_tree_source = randomtree(rng=rng)
\end{verbatim}

Now we can configure the MCMC to run a short simulation and save files with names that
will remind us to that they are from the no-polytomies run:
\begin{verbatim}
mcmc.ncycles = 2000
mcmc.sample_every = 10
mcmc.out.trees.prefix = "no_p_trees"
mcmc.out.trees.mode = REPLACE
mcmc.out.params.prefix = "no_p_params"
mcmc.out.params.mode = REPLACE
mcmc.out.log.prefix = "no_p_output"
mcmc.out.log.mode = REPLACE
\end{verbatim}

Finally, we call \cmd{mcmc()} which starts the MCMC analysis:
\begin{verbatim}
mcmc()
\end{verbatim}

After it has completed we can summarize the results:

\begin{verbatim}
sumt.outgroup_taxon = "Oedogonium cardiacum"
sumt.trees = "no_p_trees.t"
sumt.burnin = 101
sumt.out.log.prefix = "no_p_output"
sumt.out.log.mode = APPEND
sumt.out.trees.prefix = "no_p_sumt_trees"
sumt.out.trees.mode = REPLACE
sumt.out.splits.prefix = "no_p_sumt_splits"
sumt.out.splits.mode = REPLACE
sumt()
\end{verbatim}
Take a look at the summaries (they should be in the files \localfile{no\_p\_splits.pdf} and \localfile{no\_p\_trees.pdf}) and  the strength of support for the clades.

Note that a couple of the internal branches have very short lengths but 
also display very high posterior probabilities.  
\QandA{Which groupings in the tree do you think will display posterior probabilities
which are very sensitive to the prior over tree topologies?}{We are about to find out, but in most runs it the clade of:\\ {\em Heterochlamydomonas rugosa} + {\em Heterochlamydomonas inaequalis}\\ and the clade that delimited by the most recent common ancestor:\\ {\em Carteria obtusa} and {\em Volvox carteri}}

\subsection{Allowing Polytomies}
Lets redo the same analysis but allow for polytomies.

Rather than type every thing in again and just change a few commands, lets give \phycas a script.
We can get the last commands from the IPython session by using the command \cmd{history}:
\begin{verbatim}
history
\end{verbatim}
Then you can copy the output and paste it into a file.  
You'll have to remove the prompts that IPython displays (and shows in the history) and then save the file with the ending ``.py''
Alternatively you can edit the \localfile{NoPolytomy.py} script and save it as  \localfile{Polytomy.py}.

Here are the changes that we need to make:
Before we run the file and before \cmd{mcmc()} function call:
\begin{verbatim}
mcmc.allow_polytomies = False
\end{verbatim}
That will allow the sampler to consider trees that are not fully resolved, 
but we still have to specify a prior to the different types of tree.

\phycas supports priors on the trees based such that different ``resolution classes'' of trees have a particular prior probability, but the easiest prior to understand (and one that seems to work well in practice) is to use the ``polytomy'' prior discussed by \citet{LewisHH2005}.
In this prior, you specify the ratio of the prior probability between a tree over the probability of a tree that has one fewer internal branch.
So this is the ratio associated with collapsing a branch in the tree (the ratio that favors the 
acceptance of a delete-edge form of the ``bush move'' in \phycas).
Using a value for this prior ratio of $e$ (the base of the natural logarithm) was 
suggested by \citet{LewisHH2005}.
This value has no particular significance or justification in this
context (there is no theory saying that you {\em should} use $e$),
 but it is aesthetically pleasing in that a tree with one more
branch has to be more than one log-likelihood unit better in
order to be favored over the tree with a polytomy.

We can set up this analysis by invoking:

\begin{verbatim}
mcmc.polytomy_prior = True
mcmc.topo_prior_C = 2.72
\end{verbatim}
where \cmd{mcmc.topo\_prior\_C} is the prior ratio.

Since python is a fully functional language it has functions to
calculate powers of $e$; so we could have avoided rounding 
error by using:
\begin{verbatim}
mcmc.polytomy_prior = True
import math
mcmc.topo_prior_C = math.exp(1.0)
\end{verbatim}

Let's make sure to rename our output files, so that we do not overwrite the files that we just produced
in the run that did not consider polytomies:
\begin{verbatim}
mcmc.out.trees.prefix = "polytomy_trees"
mcmc.out.params.prefix = "polytomy_params"
sumt.trees = "polytomy_trees.t"
sumt.out.log.prefix = "polytomy_output"
sumt.out.trees.prefix = "polytomy_sumt_trees"
sumt.out.splits.prefix = "polytomy_sumt_splits"
\end{verbatim}

Now we have changed the command script to run the file. Save it as \localfile{Polytomy.py}

\subsection{Executing a script}
How do we tell \phycas execute it?

The easiest way to do this (on Mac) is to simply drag the \localfile{Polytomy.py} file onto the \phycas icon (in the dock if \phycas is running or on the to the application icon if it is not running).

If you are {\bf not} running the double-clickable \phycas, then you can invoke:
\begin{verbatim}
cd <type the directory with the Polytomy.py script here>
python -i Polytomy.py
\end{verbatim}

A third way to do this is better when you already have a phycas session underway. 
From within the python (or IPython) interpreter, you can execute the command:
\begin{verbatim}
execfile("Polytomy.py")
\end{verbatim}
to execute a python script (in this case  \localfile{Polytomy.py}).

\subsection{Comparing the results}
If all goes well, you should have the files created by the run:
\localfile{polytomy\_params.p}, \localfile{polytomy\_sumt\_splits.pdf}, \localfile{polytomy\_sumt\_trees.pdf}, \localfile{polytomy\_trees.t}, and \localfile{polytomy\_sumt\_trees.tre}.
Look at the \localfile{polytomy\_sumt\_splits.pdf}  for evidence of poor mixing  during the MCMC; 
and compare the majority-rule consensus trees and the maximum a posterior trees shown in  \localfile{polytomy\_trees.pdf} to those found in \localfile{no\_p\_sumt\_trees.pdf}

\section{Different priors for internal and terminal edge lengths}
\newpage
\subsection{basic.py}\label{basicpy}
\begin{verbatim}
#!/usr/bin/env python
from phycas import *

mcmc.current()
mcmc.data_source = "green.nex"
mcmc.out.log = "basic.log"
mcmc.out.log.mode = REPLACE
mcmc.out.trees.prefix = "green"
mcmc.out.params.prefix = "green"
mcmc.ncycles = 2000
mcmc.sample_every = 10
mcmc()
sumt.trees = "green.t"
sumt()
\end{verbatim}


\newpage
\subsection{NoPolytomy.py}\label{NoPolytomy}
\begin{verbatim}
from phycas import *

file_contents = readFile('ShoupLewis.nex')
mcmc.data_source = file_contents.characters

model.type = 'hky'   
model.num_rates = 4       
model.gamma_shape_prior = Exponential(2.0)
model.base_freq_param_prior =  Gamma(1.0, 1.0)
model.kappa_prior = Exponential(.25)
model.edgelen_hyperprior = InverseGamma(2.1, 1.0/1.1)


mcmc.allow_polytomies = False

rng = Lot()
rng.setSeed(13957)

mcmc.rng = rng
mcmc.starting_tree_source = randomtree(rng=rng)

mcmc.slice_max_units = 0
mcmc.ncycles = 2000
mcmc.sample_every = 10

mcmc.out.trees.prefix = 'no_p_trees'
mcmc.out.trees.mode = REPLACE
mcmc.out.params.prefix = 'no_p_params'
mcmc.out.params.mode = REPLACE
mcmc.out.log.prefix = 'no_p_output'
mcmc.out.log.mode = REPLACE

mcmc()

sumt.outgroup_taxon = 'Oedogonium cardiacum'
sumt.trees          = 'no_p_trees.t'
sumt.burnin         = 101

sumt.out.log.prefix = 'sumtno_p_output'
sumt.out.log.mode   = APPEND
sumt()
\end{verbatim}

\newpage
\bibliography{phylo}
\end{document}

\label{executingdatafile}
\execmd{cmdhere}
\localfile{algaemb.nex} 
\cmdopt{tratiopr} 

\QandA{q?}{answer}


