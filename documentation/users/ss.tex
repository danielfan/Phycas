\subsubsection{Input options:}\begin{description}
\item[\bftt  nbetavals] The number of values beta will take on during the run; for example, if this value is 4, then beta will take on these values: 1, 2/3, 1/3, 0 (default: {\bftt  101})
\item[\bftt  scubed] If True, steppingstone sampling will be performed using a working prior fit to the posterior (this greatly improves the accuracy of steppingstone sampling and is strongly recommended); if False, steppingstone sampling will be performed without using a working prior distribution. (default: {\bftt  True})
\item[\bftt  xcycles] The number of extra cycles (above and beyond mcmc.ncycles) that will be spent exploring the posterior (additional posterior cycles help scubed analyses formulate an effective working prior). (default: {\bftt  0})
\item[\bftt  maxbeta] The first beta value that will be sampled. (default: {\bftt  1.0})
\item[\bftt  minbeta] The last beta value that will be sampled. (default: {\bftt  0.0})
\item[\bftt  minsample] Minimum sample size needed to create a split-specific edge length working prior. (default: {\bftt  10})
\item[\bftt  shape1] The first shape parameter of the distribution used to determine the beta values to be sampled. This distribution is, confusingly, a Beta distribution. Thus, if both shape1 and shape2 are set to 1, beta values will be chosen at uniform intervals from minbeta to maxbeta. (default: {\bftt  1.0})
\item[\bftt  shape2] The second shape parameter of the distribution used to determine the beta values to be sampled. This distribution is, confusingly, a Beta distribution. Thus, if both shape1 and shape2 are set to 1, beta values will be chosen at uniform intervals from minbeta to maxbeta. (default: {\bftt  1.0})
\end{description}
