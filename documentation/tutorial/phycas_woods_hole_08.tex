\documentclass{article}

\setcounter{page}{1}
\setcounter{secnumdepth}{0}
\setlength{\oddsidemargin}{0.in}
\setlength{\evensidemargin}{0.in}
\setlength{\textwidth}{6.5in}
\setlength{\topmargin}{0.0in}
\setlength{\headheight}{0.17in}
\setlength{\headsep}{0.35in}
\setlength{\textheight}{8.in}
\setlength{\parindent}{0in}
\setlength{\parskip}{.1in}
%\input seteps

\usepackage{xspace}
\newcommand{\barLone}{\bar{L}_1}
\newcommand{\barLzero}{\bar{L}_0}
\newcommand{\execmd}[1]{\texttt{#1}\\}
\newcommand{\cmdopt}[1]{\texttt{#1}\xspace}
\newcommand{\cmd}[1]{\texttt{#1}\xspace}
\newcommand{\mb}{MrBayes\xspace}
\newcommand{\paup}{PAUP*\xspace}
\newcommand{\phycas}{Phycas\xspace}
\newcommand{\localfile}[1]{\textsf{#1}\xspace}
\newcommand{\QandA}[2]{\textit{#1}\footnote{#2}\xspace}
% use the fancyhdr package to allow both headers and footers
\usepackage{fancyhdr}
\pagestyle{fancy}
\fancyhf{} % clear all headers and footers
\fancyhead[LO,RE]{Woods Hole}
\fancyhead[RO,LE]{Workshop on Molecular Evolution}
\fancyfoot[LO,LE]{{\footnotesize \copyright\ 2008 by Paul O. Lewis, Mark T. Holder, and David Swofford}}
\fancyfoot[CE,CO]{}
\fancyfoot[RO,RE]{\thepage}

\usepackage{url}

\usepackage{hyperref}
\hypersetup{backref,  pdfpagemode=FullScreen,  linkcolor=blue, citecolor=red, colorlinks=true, hyperindex=true}
\fancyhead[CE,CO]{\phycas Tutorial}
\begin{document}

%###########################################################################################
\section{\phycas Lab}

%###########################################################################################

The main goal is to become familiar with running \phycas and interpreting its output. 
Don't rush things. 
If you don't finish the lab, you can always download 
\phycas from \url{http://www.phycas.org} when you get home and finish it there.

Stylistic conventions used in this document:

\begin{tabular}{l}
	Commands understood by the \paup and/or \mb programs are in \cmd{this fixed width font} \\
	Questions for you to answer are in {\em italics}. \\
	Important things to note are in {\bf bold face}. \\
	Names of files \localfile{are in  this font}.\\
	Web site URLs look \url{http://like_this} and are clickable links.
\end{tabular}

%-------------------------------------------------------------------------------------------
\subsection{Background}
%-------------------------------------------------------------------------------------------
\phycas is a software package for Bayesian phylogenetic analysis written by Paul Lewis with
some contributions by Mark Holder and David Swofford.
\phycas is a hybrid program: computationally-intensive tasks are implemented in C++, and the high-level
tasks are done in \href{http://www.python.org/}{python}\footnote{The Phycas manual contains instructions for installing python if you are using an operating system
that does not come with python installed.}.

In fact, the python language is the front-end for \phycas. 
So when you write command file in \phycas you are actually writing a python program.
This has advantages because python interpreter is a powerful, robust, and versatile program for interpreting text commands.
However, it also means that \phycas cannot be forgiving when it comes to the structure of commands and syntax in general.

\subsubsection{The Phycas.app bundle}
In the lab we will be running \phycas as a double-clickable program that is built around the open-source iTerm Terminal application (\href{http://iterm.sourceforge.net/}{iTerm homepage}).
You can run phycas scripts from any terminal by invoking python and entering the command:
\begin{verbatim}
from phycas import *
\end{verbatim}
You will not have to issue this command when using the \phycas application bundle on Mac, but you will have to use the command in all of your phycas scripts.
\subsubsection{IPython}
In the lab we will be running \phycas through \href{http://ipython.scipy.org/moin/}{IPython} rather than Python itself.  
IPython provides a more user-friendly environment for interactive python than the ``raw'' python interpreter.
The main benefits of IPython are:
\begin{itemize}
	\item Tab-completion of variable names (start to type the name and then hit the Tab-key to fill in the rest of the word or see a list of possible matches),
	\item Native support for a few UNIX commands (most notably \cmd{ls} and \cmd{cd})
	\item Support for any UNIX commands if you start the line with an exclamation point (!)
\end{itemize}
If you decide that you do not like IPython, then you can simply edit the active configuration file of phycas. 
Open 
\localfile{$\sim$/.phycas/active\_phycas\_env.sh} 
in a text editor, and change the last line to read:
\begin{verbatim}
export USES_I_PYTHON=0
\end{verbatim}

If you are unsure whether or not you have correctly enabled (or disabled) the IPython-style of interacting with \phycas, you can tell by looking at the prompt.
The IPython prompt looks like this:
\begin{verbatim}
In [2]: 
\end{verbatim}
while the python prompt looks like this:
\begin{verbatim}
>>>
\end{verbatim}


\subsection{Exercise 1}
Locate the directory of example files labeled \localfile{PhycasWHExamples} and move it to a convenient location for you.

You can launch the \phycas application you can double-click the \phycas icon application (later when you have scripts written, you can launch the \phycas by dragging a python script onto the \phycas application icon).
After the libraries have loaded you will see a prompt.
This is the python interpreter waiting for you to give it a command.

The first step is to set the working directory of \phycas to the directory in which you want to work.
If you are using the IPython interface, and you would like to  move to a directory called \localfile{Desktop/PhycasWHExamples} then issue the command:
\begin{verbatim}
cd Desktop/PhycasWHExamples
pwd
\end{verbatim}
to change working directory and then show the path to the current directory

Note that, if you are using the ``raw'' python interface, you have to issue the commands:
\begin{verbatim}
import os
os.chdir("Desktop/PhycasWHExamples")
from phycas import *
\end{verbatim}


To read in the data into memory we invoke the \cmd{readFile} function.
We pass in the full path to the NEXUS file that we would like to read as an argument to the function, and we capture a reference to the data that has been read by storing the return value of the function.
If we would like to execute the example data file ``green.nex'' and store the results in a variable called 
\texttt{file\_contents} then we use the command:
\begin{verbatim}
file_contents = readFile("green.nex")
\end{verbatim}


\begin{verbatim}
mcmc.data_source = 'green.nex' #file_contents.characters
mcmc.out.log = 'basic.log'
mcmc.out.log.mode = REPLACE
mcmc.out.trees.prefix = 'green'
mcmc.out.params.prefix = 'green'
mcmc.ncycles = 2000
mcmc.sample_every = 10
mcmc()
\end{verbatim}




\end{document}

\label{executingdatafile}
\execmd{cmdhere}
\localfile{algaemb.nex} 
\cmdopt{tratiopr} 

\QandA{q?}{answer}


