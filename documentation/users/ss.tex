\paragraph[Input options]{Input options:}\begin{description}
\item[\bftt  nbetavals] The number of values beta will take on during the run; for example, if this value is 4, then beta will take on these values: 1, 2/3, 1/3, 0 (default: {\bftt  101})
\item[\bftt  ti] If True, the marginal likelihood will be estimated using thermodynamic integration and the stepping stone method with reference distribution equal to the prior; if False (the default), the stepping stone method with reference distribution approximating the posterior will be used (this greatly improves the accuracy of the stepping stone method and is strongly recommended). (default: {\bftt  False})
\item[\bftt  xcycles] The number of extra cycles (above and beyond mcmc.ncycles) that will be spent exploring the posterior (additional posterior cycles help stepping stone analyses formulate an effective reference distribution). (default: {\bftt  0})
\item[\bftt  maxbeta] The first beta value that will be sampled. (default: {\bftt  1.0})
\item[\bftt  minbeta] The last beta value that will be sampled. (default: {\bftt  0.0})
\item[\bftt  minsample] Minimum sample size needed to create a split-specific edge length working prior. (default: {\bftt  10})
\item[\bftt  shape1] The first shape parameter of the distribution used to determine the beta values to be sampled. This distribution is, confusingly, a Beta distribution. Thus, if both shape1 and shape2 are set to 1, beta values will be chosen at uniform intervals from minbeta to maxbeta. (default: {\bftt  1.0})
\item[\bftt  shape2] The second shape parameter of the distribution used to determine the beta values to be sampled. This distribution is, confusingly, a Beta distribution. Thus, if both shape1 and shape2 are set to 1, beta values will be chosen at uniform intervals from minbeta to maxbeta. (default: {\bftt  1.0})
\end{description}
