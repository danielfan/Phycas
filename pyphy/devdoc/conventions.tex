\documentclass[10pt]{article}
\setlength{\oddsidemargin}{0.in}
\setlength{\textwidth}{6.5in}
\setlength{\topmargin}{-0.25in}
\setlength{\textheight}{8.25in}

% Skip space between paragraphs
\setlength{\parskip}{.1in}

% Do not put numbers in section headings
\setcounter{secnumdepth}{0}

% Indent paragraphs 0. in
\setlength{\parindent}{0.in}

% Use the natbib package for the bibliography
\usepackage[round]{natbib}
\bibliographystyle{sysbio}

% Use the graphicx package to incorporate and scale
% encapsulated postscript figures
\usepackage{graphicx}

% Make document single-spaced
\renewcommand{\baselinestretch}{1.0}

\newcommand{\svnurl}{https://svn.scs.fsu.edu/svn/phycasdev}
\newcommand{\None}{\verb+None+}

\begin{document}

\title{Phycas Policies and Conventions}
\author{Paul O. Lewis} 
\date{16 May 2006}
\maketitle

\section{Conventions and Policies}

Here are some conventions and policies that I have established, along with my rationale for each. Please feel free to argue with the rationale if you disagree with the policy or convention, and I'll be happy to change both as long as we can all come to an agreement.

\begin{description}
\item[Convention: major and minor release numbering] Releases are labeled as follows: {\tt rel\_0\_1\_2}. This example specifies the 0th ``major'' release, the 1st ``minor'' release, and the 2nd ``bugfix'' release. In general, if new features are added, this should result in a major or minor release, with bugfix releases just repairing problems that arise after the last minor release. All levels (major, minor and bugfix) range from 0 to infinity (i.e. they need not stop at 9) 

{\em {\bfseries Rationale:} seems sensible, but after beginning to use SVN it might make more sense to change this to simply use the SVN master revision number.}

\item[Policy: test before commit] The script {\tt runall.bat} (or {\tt runall.sh}) in the {\tt phycasdev/pyphy/apps} directory should be run successfully before committing a new major or minor release. Also, {\tt doctestall.py} in {\tt phycasdev/pyphy/pyphy} should be run successfully to test all the examples embedded in Python files. 

{\em {\bfseries Rationale:} sometimes it is very easy to make a change that affects the output of the examples (a single new draw from Lot::Uniform() will do it!), and the more divergent the source code becomes, the harder it is to identify the change that caused the example to begin generating output different from its reference output.}

\item[Policy: documentation duplication] Functions implemented in C++ and exported to Python should be provided with two sets of documentation comments. The C++ documentation is probably going to be seen only by dedicated programmers and can be very technical and aimed at people who will extend Phycas, whereas the version in the Python source code will be shown to users who use the Python help function and thus should be aimed at a more general audience, should provide doctestable examples, and explain how to use the function. 

{\em {\bfseries Rationale:} this policy provides for useful online help for users as well as ensuring that we can understand each other's C++ code.}

\item[Convention: Python source code comments] Here is an example of a typical documentation comment in Python code:
\begin{verbatim}
    def getMean(self):
        #---+----|----+----|----+----|----+----|----+----|----+----|----+----|
        """
        Returns the mean of the distribution. This is the theoretical mean
        (i.e., it will not change if sample is called to generate samples
        from this distribution).

        >>> from ProbDist import *
        >>> b = ExponentialDist(2)
        >>> print b.getMean()
        0.5

        """
        return ExponentialDistBase.getMean(self)
\end{verbatim}
The Python ``ruler'' comment just after the first line of the function serves as a length guide (it is 70 characters long). The documentation comment itself is set off by triple quotes. The last line of the comment should be blank (this is necessary if a doctest example is the last item). This documentation comment has a doctestable example that illustrates how to use the function (examples should be inserted whereever possible).

 {\em {\bfseries Rationale:} ensures consistency and the line length limit ensures that most users will not see text that has lines broken in frustrating places.}
%\item[Policy: XXXXXXXXXX] XXXXXXXXXX {\em {\bfseries Rationale:} XXXXXXXXX.}
%\item[Convention: XXXXXXXXXX] XXXXXXXXXX {\em {\bfseries Rationale:} XXXXXXXXX.}
\end{description}

%
% Figure "felsenstein"
%
%\clearpage
%\begin{figure}
%\centering
%\hfil\includegraphics[scale=0.7]{felsenzone.eps}\hfil
%\end{figure}

%\section{Literature Cited}
%\renewcommand{\bibsection}{}
%\bibliography{devsummary}

\end{document}
