\chapter{Parameters and moves} 

A \term{move} in \phycas\ involves a Metropolis-Hastings update mechanism, whereas a \term{parameter} uses slice sampling.

%%%%%%%%%%%%%%%%%%%%%%%%%%%%%%%%%%%%%%%%%%%%%%%%%%%%%%%%%%%%%%
%%%%%%%%%%%%%%%%%%%%%%%%%%%%%%%%%%%%%%%%%%%%%%%%%%%%%%%%%%%%%%
\section{Adding a new move}
%%%%%%%%%%%%%%%%%%%%%%%%%%%%%%%%%%%%%%%%%%%%%%%%%%%%%%%%%%%%%%
%%%%%%%%%%%%%%%%%%%%%%%%%%%%%%%%%%%%%%%%%%%%%%%%%%%%%%%%%%%%%%

\begin{enumerate}
\item Create header (*.hpp) and source code (*.cpp) files for your move. The easiest way to do this is to copy the files for an existing move. Here is the basic boilerplate code for the header file:
\begin{verbatim}
#if ! defined(MY_NEW_MOVE_HPP)
#define MY_NEW_MOVE_HPP

#include <boost/weak_ptr.hpp>		// for boost::weak_ptr
#include "phycas/src/mcmc_updater.hpp"	// for base class MCMCUpdater

class MCMCChainManager;
typedef boost::weak_ptr<MCMCChainManager>  ChainManagerWkPtr;

namespace phycas
{

/*----------------------------------------
|   Description of what MyNewMove does.
*/
class MyNewMove : public MCMCUpdater
    {
    public:
                MyNewMove();
        virtual ~MyNewMove();
        bool    update();
    };
}  // namespace phycas
\end{verbatim}
And here is the corresponding minimal boilerplate for the source code file:
\begin{verbatim}
namespace phycas
{
/*----------------------------------------
|   MyNewMove constructor.
*/
MyNewMove::MyNewMove()
  {
  }

/*----------------------------------------
|   MyNewMove destructor.
*/
MyNewMove::~MyNewMove()
  {
  }

/*----------------------------------------
|   Description of update method.
*/
bool NielsenMappingMove::update()
    {
    // code for performing the update goes here
    return true;
    }
} // namespace phycas

\end{verbatim}
\item Write the constructor, destructor, and \method{update} member function, creating helper methods as necessary. The \method{update} method should return true if it changes the model, false if it leaves everything untouched (e.g. a Metropolis-Hastings proposal that is rejected). Note that because \class{MyNewMove} is derived from \class{MCMCUpdater}, it has access to the following shared pointers (as well as the other public and protected data members of the \class{MCMCUpdater} class):
\begin{itemize}
\item \shptr{tree} points to the \class{Tree} object
\item \shptr{tree\_manipulator} points to the \class{TreeManip} object
\item \shptr{model} points to the \class{Model} object
\item \shptr{likelihood} points to the \class{TreeLikelihood} object
\item \shptr{rng} points to the \class{Lot} (pseudorandom number generator) object
\end{itemize}
\item In \pathname{updater\_pymod.cpp}, add a Boost Python entry so that your new move can be called from Python code:
\begin{verbatim}
#include "phycas/src/my_new_move.hpp"
...
class_<phycas::MyNewMove, bases<phycas::MCMCUpdater>, 
  boost::noncopyable, boost::shared_ptr<phycas::MyNewMove> >("MyNewMove") 
  .def("update", &phycas::MyNewMove::update)
  ;
\end{verbatim}
\item Add the new move to the \method{setupChain} function of \class{MCMCManager}
\end{enumerate}

%%%%%%%%%%%%%%%%%%%%%%%%%%%%%%%%%%%%%%%%%%%%%%%%%%%%%%%%%%%%%%
%%%%%%%%%%%%%%%%%%%%%%%%%%%%%%%%%%%%%%%%%%%%%%%%%%%%%%%%%%%%%%
\section{Larget-Simon LOCAL move}
%%%%%%%%%%%%%%%%%%%%%%%%%%%%%%%%%%%%%%%%%%%%%%%%%%%%%%%%%%%%%%
%%%%%%%%%%%%%%%%%%%%%%%%%%%%%%%%%%%%%%%%%%%%%%%%%%%%%%%%%%%%%%

%
% Figure "MBTA"
%
%\clearpage
%\begin{figure}
%\centering
%\includegraphics[scale=0.5]{Figures/MBTA.eps}
%\label{MBTA}
%\caption{\small Map of the Massachusetts Bay Transit Authority subway system is a good example of a model. Is is an abstraction of reality that is nevertheless much more useful (for the purpose of using the subway system) than a very realistic map of the region.}
%\end{figure}

